
\chapter{总结}

本毕设完成了在AMD64多核平台下,针对操作系统上的多核可扩展性问题,进行了理论上的分析,并完成了uCore多核支持和相应的性能测试工具。具体来说,本毕设所做的工作如下:

\begin{itemize}
	\item
	调研了研究界和工业界在操作系统多核可扩展性上所做的工作,为本系统的实现奠定了基础
	\item
	介绍了对操作系统虚拟内存子系统接口(包括系统调用和硬件中断)建模的方法,并提出了使用KLEE对接口模型做可交换性求解的方法
	\item 在实验操作系统uCore上实现了多核硬件无关层模型和AMD64 NUMA多核硬件平台的支持
	\item
	在上一点实现的硬件无关层上,实现了无锁堆栈、可扩展percpu变量、Refcache引用计数器和NUMA物理页分配器等多核可扩展操作系统基础设施
	\item 提出了一种新的基于QEMU模拟器的性能测试工具QProf
\end{itemize}

本毕设所提出的用KLEE求解系统接口可交换性的方法具有一定的研究价值。另一方面,本毕设在uCore上实现的多核架构基础设施,为以后的多核操作系统可扩展性研究和教学奠定了基础。由于本毕设设计的uCore
AMD64平台支持不但可以在真实机器上运行,还可以在QEMU和KVM模拟器上运行,大大降低内核调试的难度。最后,本毕设提出的QProf性能测试工具,提供了一种简便的全系统性能测试手段,具有较大的应用价值。

从另一方面看,本毕设的操作系统建模方法和uCore多核操作系统还有进一步完善的空间。首先,本毕设设计的虚拟内存系统模型并没有考虑文件映射内存页和进程见共享内存等情况。为了求解这些情况下的系统接口的交换性,必须对操作系统内核中其他子系统进行建模。其次,尽管本毕设实现了多核实验性操作系统原型uCore
AMD64,还需要大量的工作才能使其支持Metis\cite{linux:osdi10}等复杂应用。

