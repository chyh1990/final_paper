
%%% Local Variables:
%%% mode: latex
%%% TeX-master: t
%%% End:


\chapter{操作系统性能分析工具设计实现}

无论在程序的开发还是真实使用的过程中,发现和调试系统中出现的性能问题都是至关重要的。用户程序的性能优化迭代流程一般是这样的:
\begin{enumerate}
\item 使用性能采集工具(如GProf、Intel VTune等)对目标程序进行测试;
\item 通过测试结果,分析软件运行的关键路径,找到性能瓶颈;
\item 对关键路径上的数据结构和算法做优化,如效果不理想跳到1;
\item
当数据结构和算法都优化到最优状态后,进行体系结构相关优化,如Cache使用优化、使用内联汇编等。在此期间不断测试。
\end{enumerate}

从上述的软件优化流程可以看到,性能采集工具在优化过程中占有至关重要的作用。为此,开源软件界和工业界都开发了大量用户态性能测试工具,如开源的gprof和perf工具,和商用的Intel
VTune等。但是对于操作系统内核的有效性能分析工具却不多。虽然在Linux内核中集成了perf和ftrace性能分析子系统,这些工具都存在一些问题。最明显的是,它们与Linux内核耦合度太强,无法单独使用,因此无法对xv6、uCore等实验操作系统做性能测试。另外,这些工具运行在与内核相同的特权级别上,不能检测到系统内核的所有行为。为此,本毕设设计了基于QEMU的全系统性能采集工具QProf。本章将主要介绍QProf的工作原理。

\section{概述}
为了在模拟器中方便地寻找操作系统内核的瓶颈,本毕设这种方法结合了Qemu和gprof的两个开源软件,实现了一个全系统的Profiler
\pozhehao
QProf\footnote{https://github.com/chyh1990/qemu/tree/1.3-profiler},本系统具有以下特征:
\begin{itemize}
\item 通过模拟器,运行在比操作系统内核更高的级别上;
\item 使用了编译器的相关特性,不需对内核源代码做任何修改;
\item 生成与gprof兼容的测试报告,方便后期分析处理。
\end{itemize}

由于真实操作系统工作环境的复杂性,很难通过分析源代码等静态分析方法发现内核中的性能问题。
本全系统性能测试工具通过动态分析的方法,在内核运行过程中允许程序员方便地对操作系统内核的各个函数进行跟踪和计时。通过这些信息,可以较为容易地发现内核的运行的关键路径
(也正是内核代码的瓶颈所在),并发现潜在的性能问题。
真实应用上的测试结果第\ref{cha:test}章。

\section{QEMU模拟器架构分析}

QEMU\cite{Bellard:2005:QFP:1247360.1247401}是一个可移植的基于动态二进制翻译技术的硬件模拟器。QEMU可以模拟包括x86、PowerPC、ARM等多种CPU,并支持全系统硬件模拟,可以在这个模拟平台上运行Windows、Linux等多种操作系统。另外,在x86平台上,QEMU也支持KVM硬件虚拟化后端,大大提高了模拟器的执行性能。为了保证采样的正确性,现在QProf不支持KVM硬件虚拟化技术,但通过简单扩展可以方便实现。这里只对QEMU的软件动态二进制执行部分作讨论。

\begin{figure}
\centering
\begin{tikzpicture}

\tikzstyle{vecArrow} = [thick, decoration={markings,mark=at position
   1 with {\arrow[semithick]{open triangle 60}}},
   double distance=1.4pt, shorten >= 5.5pt,
   preaction = {decorate},
   postaction = {draw,line width=1.4pt, white,shorten >= 4.5pt}]

\tikzstyle{comp}=[draw, fill=blue!20, text width=4em, 
    text centered, minimum height=2.5em,drop shadow,node distance=5em]

\node[draw,comp] (a) {Guest};
\node[draw,comp,below of=a] (b) {QEMU};
\node[draw,comp,below of=b] (c) {Host};

\node[draw,comp,right of=b,node distance=6em] (gc) {Guest Code};
\node[draw,comp,right of=gc,node distance=6em] (tcg) {TCG};
\node[draw,comp,right of=tcg,node distance=6em] (target) {TCG target};

\draw[vecArrow] (a) to (b);
\draw[vecArrow] (b) to (c);

\draw[vecArrow] (gc) to (tcg);
\draw[vecArrow] (tcg) to (target);

\path [draw,-latex',dashed] (target) |- (c);

    \begin{pgfonlayer}{background}
        \path (b.west |- b.north)+(-0.5,0.3) node (a) {};
        \path (target.east |- target.south)+(+0.5,-0.3) node (c) {};

        \path[fill=yellow!20,rounded corners, draw=black!50, dashed]
            (a) rectangle (c);

    \end{pgfonlayer}


\end{tikzpicture}
\label{fig:qemu-arch}
\caption{QEMU模拟器系统架构}
\end{figure}

QEMU系统的架构见图\ref{fig:qemu-arch},其中的Tiny Code
Generator(TCG)是QEMU系统的核心,它把Guest的CPU指令翻译为QEMU内部的中间表示。对于普通指令,QEMU中有通用的指令实现代码。当遇到操作系统中的特权指令和涉及到协处理器、MMU操作等硬件相关指令时,QEMU将回调特定的Helper函数软件模拟硬件的行为。QProf使用这项特性,添加了一条AMD64
MSR控制寄存器的控制指令,实现被测试操作系统内核(Guest Code)对QEMU的回调。

另一方面,从QEMU的架构图中可以看出,QEMU工作在比被测试操作系统更高级的位置上。通过对QEMU进行修改,QProf可以准确地获得被测系统的全部状态。

\section{QProf工作原理}
	目前,对程序做性能测试的方法主要有两种:
	\begin{enumerate}
		\item {\heiti 插桩法}
			在程序编写、编译或链接(包括动态链接)的过程中,注入跟踪代码;
		\item {\heiti 采样法}
			系统中定时产生时钟中断时间,打断程序运行,采集当前程序计数器值(Program
			Counter)。通过采样点的分布,估计各个函数运行时间。
	\end{enumerate}

	QProf系统结合了上述两种方法对操作系统内核进行性能测试。它与传统用户态性能测试工具的主要区别是,QProf使用Qemu作为运行环境,可以跟踪CPU的底层状态。使用模拟器的跟踪操作系统内核性能情况的一个问题是,Qemu中使用软件模拟外设硬件的行为,如果操作系统的真正瓶颈在IO操作上(而非算法),则QProf难以直接发现真实的性能瓶颈。但是通过QProf采集的函数调用次数等其他信息,开发者可以分析和推测系统的问题所在。 QProf的主要原理如下所示:
	\begin{itemize}
		\item 编译内核时加入-pg参数,GCC会在所有
			函数的prolog中插入一个mcount函数调用,
			用于跟踪函数调用图(Call graph):
\begin{lstlisting}[language={[x86masm]Assembler}, caption=加入-pg编译参数后的函数序言]
<x2_cpu_init>:                                                 
	55                      push   %rbp                     
	48 89 e5                mov    %rsp,%rbp                
	48 83 ec 10             sub    $0x10,%rsp               
	e8 93 23 fb ff          callq  ffff80000022ce24 <mcount>
	48 89 7d f8             mov    %rdi,-0x8(%rbp)          
\end{lstlisting}

		\item 在内核中实现mcount函数,加入一条对Supervisor(QEMU)
			回调指令。在QProf中,使用如下汇编代表做回调:
\begin{lstlisting}[caption=QProf中mcount函数实现]
        movq    56(%rsp),%rsi
        /* Get frompc via the frame pointer.*/
        movq    8(%rbp),%rdi
        /* call the hypervisor */
        mov     $MAGIC_ECX, %ecx
        wrmsr
\end{lstlisting}
		\item 修改QEMU,对Supervisor的回调进行响应,生成调用图;
		\item QEMU中使用时钟信号定时采样,对内核做profile。
	\end{itemize}

\section{小结}

