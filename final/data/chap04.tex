
\chapter{多核操作系统和相关性能测试工具的实现}
在上一章多核操作系统可扩展性理论的基础上,本章将介绍在清华uCore实验操作系统的基础上加入对最新NUMA的多核处理器的支持,并用数种多核可扩展的数据结构和算法改造其内核数据结构。

\section{AMD64多核NUMA架构硬件抽象层的实现}
本毕设在实验操作系统内核uCore的硬件抽象层基础上,加上了相对完整的x86\_64 NUMA多核架构支持。本内核已经在QEMU软件模拟器、KVM硬件虚拟化平台和Intel真实机器上测试通过。
\subsection{uCore硬件抽象层概述}

\subsection{ACPI支持}

\subsection{多核中断和IPI中断支持}

\section{多核操作系统基础设施实现}
\subsection{锁的实现}

\subsection{支持NUMA的物理页分配器}

\subsection{per-CPU变量实现}

\section{可扩展引用技术器的实现}

\section{基于QEMU的全系统性能采集工具}
