
%%% Local Variables:
%%% mode: latex
%%% TeX-master: t
%%% End:
%\secretlevel{绝密} \secretyear{2100}

\ctitle{多核可扩展操作系统内核的实现与研究}
% 根据自己的情况选,不用这样复杂
\makeatletter
\ifthu@bachelor\relax\else
  \ifthu@doctor
    \cdegree{工学博士}
  \else
    \ifthu@master
      \cdegree{工学硕士}
    \fi
  \fi
\fi
\makeatother


\cdepartment[计算机]{计算机科学与技术系}
\cmajor{计算机科学与技术}
\cauthor{陈宇恒} 
\csupervisor{陈渝~副教授}
% 如果没有副指导老师或者联合指导老师,把下面两行相应的删除即可。
%\cassosupervisor{陈文光教授}
%\ccosupervisor{某某某教授}
% 日期自动生成,如果你要自己写就改这个cdate
%\cdate{\CJKdigits{\the\year}年\CJKnumber{\the\month}月}

% 博士后部分
% \cfirstdiscipline{计算机科学与技术}
% \cseconddiscipline{系统结构}
% \postdoctordate{2009年7月——2011年7月}

\etitle{TODO: Implementation and Analysis of Operating System on Multicore
System} 
% 这块比较复杂,需要分情况讨论:
% 1. 学术型硕士
%    \edegree:必须为Master of Arts或Master of Science(注意大小写)
%              “哲学、文学、历史学、法学、教育学、艺术学门类,公共管理学科
%               填写Master of Arts,其它填写Master of Science”
%    \emajor:“获得一级学科授权的学科填写一级学科名称,其它填写二级学科名称”
% 2. 专业型硕士
%    \edegree:“填写专业学位英文名称全称”
%    \emajor:“工程硕士填写工程领域,其它专业学位不填写此项”
% 3. 学术型博士
%    \edegree:Doctor of Philosophy(注意大小写)
%    \emajor:“获得一级学科授权的学科填写一级学科名称,其它填写二级学科名称”
% 4. 专业型博士
%    \edegree:“填写专业学位英文名称全称”
%    \emajor:不填写此项
\edegree{TODO DEGREE} 
\emajor{Computer Science and Technology} 
\eauthor{Chen Yuheng} 
\esupervisor{Associate Professor Chen Yu} 
%\eassosupervisor{Chen Wenguang} 
% 这个日期也会自动生成,你要改么?
% \edate{December, 2005}

% 定义中英文摘要和关键字
\begin{cabstract}
	随着微电子技术的迅猛发展,AMD64多核架构已经广泛应用到服务器、高性能计算等领域。各个芯片厂商不断推出新的微处理器产品,32核以上的基于AMD64多核架构的硬件平台对操作系统内核的多核可扩展性性能提出了越来越高的要求。如何使用新方法提高操作系统内核的多核可扩展性性能,已经成为备受内核开发者和操作系统研究者关注的重要课题,具有较高的研究价值和现实意义。

	本文对研究了现有开源操作系统Linux虚拟内存管理系统的多核竞争瓶颈进行了研究,提出了一种针对对象引用计数器和简化Posix标准的操作系统虚拟内存管理子系统的建模方法。并进一步提出了结合KLEE符号执行引擎和可交换性原理求解操作系统模型中可扩展实现存在条件的新方法。在此理论基础上,本文实现了一个支持AMD64
	NUMA多核架构的小型操作系统内核。最后实现了一款基于QEMU模拟器的全系统性能分析工具QProf,用于方便地测试操作系统内核性能。

\end{cabstract}

\ckeywords{操作系统, 符号执行, 多核可扩展性}

\begin{eabstract} 
	
	As the development of microelectronic technology, AMD64 multicore multiprocessor architecture has been widely used in the fields of servers, high performance
	computing and etc. Many processor manufacturers release their new chips every year, and multicore scalability on hardware systems containing more than 32 x86 cores has become a big challenge to the operating system developers. Currently,  designing new mechanism to improve the multicore scalability is an attractive 
	topic in both industry and academia.
	
	This paper analyzes the scalability bottleneck in the virtual memory subsystem of Linux, and propose models of reference counter and virtual memory management
	subsystem conforming to a simplified Posix specification. Furthermore, we combines KLEE symbolic execution engine and the commutativity rule 
	to prove the existence of multicore scalable implementations of system interfaces. Based on the theory, we implements a modern operating system kernel supporting
	AMD64 NUMA multicore  architecture. Finally, a QEMU-based full system profiler QProf is developed to find potential performance problems in the kernel.
\end{eabstract}

\ekeywords{Operating System, Symbolic Execution, Scalability}
