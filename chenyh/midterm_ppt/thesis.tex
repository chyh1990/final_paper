\documentclass{beamer}

\usecolortheme[named=blue]{structure}

\mode<presentation>
{
  \usetheme{Warsaw}
  \setbeamercovered{transparent}
  \setbeamertemplate{items}[ball]
  \setbeamertemplate{theorems}[numbered]
  \setbeamertemplate{footline}[frame number]
}

\usepackage{beamerthemesplit}
\usepackage{hyperref}
\XeTeXlinebreaklocale "zh"
\XeTeXlinebreakskip = 0pt plus 1pt
\usepackage[cm-default]{fontspec}
\usepackage{xunicode,xltxtra}
\usepackage{indentfirst}
\usepackage{xeCJK}
\usepackage{graphicx}
\usepackage{algorithm}
\usepackage{algpseudocode}
\usepackage{amsmath}
\usepackage{xcolor}
\usepackage{listings}
\usepackage{fancyvrb}
\usepackage{tabularx}

\setsansfont{Kai} 

\XeTeXlinebreaklocale "zh"
\XeTeXlinebreakskip=0pt plus 1pt minus 0.1pt


\title
{多核可扩展操作系统内核的实现与研究}
\subtitle{期中报告}
  \author{计92 陈宇恒\\指导老师:陈渝}
\institute{清华大学计算机系}

\begin{document}
%-----------------------------------------------------------------------------80
  \frame
  {
    \titlepage
  }

\section*{大纲}
\frame{\tableofcontents }

\frame{
	\frametitle{已有操作系统的问题}
  当前基于AMD64/X86\_64架构的多核系统已经得到广泛应用。但随着40核甚至
  80核的NUMA系统的出现,对已有操作系统的多核可扩展性提出了苛刻的要求。

	\begin{itemize}
		\item
	由于已有操作系统多数是由单核系统演化而来,在设计之初就使用了一些
	不具有多核扩展性的数据结构来维护VMA、DCACHE、MountTable等内核关键
	数据结构。这导致Linux等操作系统在20核以上硬件系统上表现不佳,甚至
	出现性能下降;
		\item 一些同步原语没有为重核架构优化,例如Linux的Spinlock
			实现在重核架构上导致代价高昂的Cache同步协议开销;
		\item
			操作系统设计上的问题。例如Linux中各个Core共享页表,
			在一些应用下造成不必要的TLB shootdown。
	\end{itemize}
	鉴于这些问题,我们希望重新设计一个小型实验性操作系统和相应工具,
	从根本上解决这些问题。

}


\frame{
	\frametitle{期中工作成果}
	经过几周的努力,已经完成了以下几项多核OS相关工作:
	\begin{itemize}
		\item OS
			kernel方面:完成了uCore多核启动,在QEMU、KVM和真机上启动
			成功,为后面工作奠定了基础;

		\item
			OS多核可扩展性理论研究方面:使用MIT的Spec符号执行引擎,
			编写了虚拟内存的Posix Model。
		\item 性能工具方面:修改QEMU,为其添加全系统profiler功能;
	\end{itemize}

}

\section{支持NUMA的多核OS}
\begin{frame}[fragile]
	\frametitle{ACPI系统拓扑信息读取}
	在主板出厂前,硬件厂商把硬件的拓扑结构信息写入ROM中的
	ACPI表。OS通过读取此表,获得CPU个数,NUMA节点等信息。
	使用开源实现ACPICA,并实现其操作系统相关接口层,即可实现LAPIC、IOAPIC、NUMA等信息。
	\scriptsize{
	\begin{verbatim}
acpitables_init()
ACPI: RSDP 0xfd7d0 00014 (v0 BOCHS )
ACPI: RSDT 0x1fffd370 00038 (v1 BOCHS  BXPCRSDT 00000001 BXPC 00000001)
ACPI: FACP 0x1fffff80 00074 (v1 BOCHS  BXPCFACP 00000001 BXPC 00000001)
...
ACPI: HPET 0x1fffd500 00038 (v1 BOCHS  BXPCHPET 00000001 BXPC 00000001)
ACPI: SRAT 0x1fffd3b0 00150 (v1 BOCHS  BXPCSRAT 00000001 BXPC 00000001)
xapic_init_once: xapic base 0xffff8000fee00000
xapic_lapic_init: Using xapic LAPIC
xapic: Initializing LAPIC (CPU 0)
	\end{verbatim}
}

\end{frame}

\begin{frame}[fragile]
	\frametitle{per-CPU变量和支持NUMA的物理页分配算法}
	\begin{itemize}
		\item per-CPU变量,实现原理比较简单,在kernel
			image中分配好Boot CPU的perCPU内存,其他CPU
			(Application
			Processor)动态分配。每个CPU使用GS寄存器存放各自perCPU变量指针。
		\item
			支持NUMA的物理页分配算法。由于NUMA架构上,
			每个CPU应该尽量使用本地节点的内存页,这就需要对传统Buddy
			System算法做修改,具体是对每个NUMA节点都建立各自的FreeList。
	\end{itemize}
	\scriptsize{
	\begin{Verbatim}[frame=single]
acpi: NUMA node 0 : cpus 0 2 4 6 
  0x0 - 0x9ffff
  0x100000 - 0xfffffff
acpi: NUMA node 1 : cpus 1 3 5 7 
  0x10000000 - 0x1fffffff
	\end{Verbatim}
}
		
\end{frame}

\frame{
	\frametitle{AP启动和简易调度}
	\begin{itemize}
		\item 系统上电后,只有Boot CPU开始工作,Boot
	CPU负责在OS启动的最后阶段唤醒Application Processor
	(AP)。这项工作根据Intel Multiprocessor规范,通过LAPIC发送
	Startup中断实现。

		\item Boot CPU把所有AP唤醒后,打开本地中断,开始正常运行。现在系统
	中实现了一个简单的调度器,每个CPU从一个Ready Task队列中按照Round-Robin
	顺序取出Task运行。
	\end{itemize}
}

\frame{
	\frametitle{真机测试环境}
	为了测试系统性能,并且在真实环境下深入研究Cache Contention,
	TLB shootdown等对多核OS性能的影响,我们的系统必须支持在
	真实机器上运行。

	我们使用的一台4核Core2机器作为测试机,通过PXE+Grub2网络启动
	我们的OS kernel。Root Filesystem暂时使用Ramdisk,驻留在内存。
}

\section{OS多核可扩展性理论研究}
\frame{
	\frametitle{MIT对OS syscall多核可扩展性的研究}
	MIT的最新一篇论文通过研究,认为Syscall的可交换性在一定程度上
	等价于Syscall的多核可扩展性。

	\begin{block}

	具体论证是,当两个Syscall在一定条件下可交换,意味着他们没有数据
	相关性,通过某种实现可以避免Cache Contention。达到Perfect
	Scalability。

	\end{block}

	为了发现Syscall可交换条件,需要遍历Syscall实现中的所有条件
	分支,这可以用符号执行技术实现。但是真实Syscall实现过于复杂,
	我们可以通过归纳Posix标准中对某个Syscall语义的要求,用简化
	的模型(Model)描述此语义要求。
}

\frame{
	\frametitle{Posix接口虚拟内存相关syscall的建模}
	这里只针对Anonymous Page经行建模(先不考虑File Backed MMAP)
	通过对Posix标准的阅读,结合主流OS对虚拟内存的实现机制,
	我们整理出OS虚拟内存的几个主要接口:
	\begin{itemize}
		\item mmap(addr, fixed, canwrite)
		\item munmap(addr)
		\item pagefault\_read(addr)
		\item pagefault\_write(addr)
	\end{itemize}

}

\begin{frame}[fragile]
	\frametitle{MMAP系统调用建模例子}
	\begin{columns}
	\begin{column}{0.7\textwidth}
	\tiny{
		\begin{Verbatim}[frame=single]
def mmap(self, addr, canwrite, fixed, internal_ret_addr):
  self.add_addrvar(simsym.unwrap(addr).decl())
  simsym.assume(internal_ret_addr>=0)
  simsym.assume(internal_ret_addr<PG_NUM)
  if fixed:
    internal_ret_addr = addr
  else:
    simsym.assume(simsym.symand([internal_ret_addr>0,
      internal_ret_addr<PG_NUM,
      self.vma[internal_ret_addr] == 0]))
  self.add_addrvar(simsym.unwrap(internal_ret_addr).decl())
  # never return null
  if internal_ret_addr <= 0 or internal_ret_addr > PG_NUM-1:
    return ('err', -1)
  if canwrite:
    self.vma[internal_ret_addr] = 2
  else:
    self.vma[internal_ret_addr] = 1
  return ('ok', internal_ret_addr)
    \end{Verbatim}
	}
  	\end{column}

  	\begin{column}{0.3\textwidth}
		这类考虑了Posix标准中的一些标志位:
		\begin{itemize}
			\item MAP\_FIX
			\item MAP\_ANON
			\item PROT\_WRITE
		\end{itemize}
  	\end{column}
	\end{columns}
\end{frame}

\section{基于QEMU和GPROF的全系统Profiler}
\frame{
	\frametitle{QEMU+GPROF基本原理}
	这种方法结合了QEMU和GPROF的机制,实现了一个全系统的Profiler:
	\begin{enumerate}
		\item 编译内核时加入-pg参数,GCC会在所有
			函数的prolog中插入一个mcount函数调用,
			用于跟踪Call graph
		\item 在内核中实现mcount函数,加入一条supervisor(QEMU)
			Callback指令
		\item 修改QEMU,对Callback进行响应,生成调用图。
			并定时采样,做profile。
	\end{enumerate}
	代码放在https://github.com/chyh1990/qemu/tree/1.3-profiler
}

\begin{frame}[fragile]
	\frametitle{具体实现}
	\begin{columns}[T]
	\begin{column}{0.5\textwidth}
	QEMU的更改主要包含两部分:
	\begin{itemize}
		\item glibc中移植的Callgraph和统计代码
		\item supervisor中的callback,现在的实现:
	\tiny{
		\begin{verbatim}
/* target-i386/misc_helper.c */
case __MSR_KPROFILER:
    _kmcount_internal(EDI, ESI);
    break;		
	\end{verbatim}
	}
	\end{itemize}
	\end{column}
	\begin{column}{0.5\textwidth}
OS 中的修改主要是增加mcount函数:
\tiny{
	\begin{Verbatim}[frame=single]
.global mcount
.type mcount, @function
#define MAGIC_ECX 0xfeab0001
mcount:
        /* Allocate space for 7 registers.*/
        subq    $56,%rsp
        movq    %rax,(%rsp)
        /* ... */

        movq    56(%rsp),%rsi
        /* Get frompc via the frame pointer.*/
        movq    8(%rbp),%rdi
        /* call the hypervisor */
        mov     $MAGIC_ECX, %ecx
        wrmsr

        /* ... */
        movq    (%rsp),%rax
        addq    $56,%rsp

        ret
	\end{Verbatim}
}
	
	\end{column}
	\end{columns}

\end{frame}

\begin{frame}[fragile]
	\frametitle{输出结果}
	\tiny{
	\begin{Verbatim}[frame=single]
Flat profile:

Each sample counts as 0.01 seconds.
  %   cumulative   self              self     total           
 time   seconds   seconds    calls  ms/call  ms/call  name    
 25.62      0.52     0.52     8654     0.06     0.06  ide_wait_ready
  8.87      0.70     0.18      960     0.19     0.75  ide_read_secs
  8.37      0.87     0.17                             do_halt
  4.43      0.96     0.09     1195     0.08     0.08  memmove
  4.19      1.05     0.09   135219     0.00     0.00  get_pte
  2.96      1.11     0.06     7908     0.01     0.01  serial_putc_sub
  2.71      1.16     0.06   135268     0.00     0.00  get_pmd
  1.97      1.20     0.04   433010     0.00     0.00  ptep_present
  1.97      1.24     0.04      673     0.06     0.06  memset
	\end{Verbatim}
	}

	\tiny{
	\begin{Verbatim}[frame=single]
granularity: each sample hit covers 2 byte(s) for 0.49% of 2.03 seconds

index % time    self  children    called     name
                0.00    1.25    4377/4377        __alltraps [2]
[1]     61.4    0.00    1.25    4377         trap [1]
                0.00    1.24    4377/4377        trap_dispatch [3]
                0.01    0.00   26695/26889       mycpu [157]
                0.00    0.00    8746/8755        trap_in_kernel [365]
                0.00    0.00    4362/4362        kern_enter [376]
                0.00    0.00    4355/4362        kern_leave [377]
                0.00    0.00    4278/4285        may_killed [379]
                0.00    0.00      79/117         schedule [429]
	\end{Verbatim}
	}
\end{frame}


\end{document}

