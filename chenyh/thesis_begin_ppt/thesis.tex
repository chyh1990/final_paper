\documentclass{beamer}

\usecolortheme[named=blue]{structure}

\mode<presentation>
{
  \usetheme{Warsaw}
  \setbeamercovered{transparent}
  \setbeamertemplate{items}[ball]
  \setbeamertemplate{theorems}[numbered]
  \setbeamertemplate{footline}[frame number]
}

\usepackage{beamerthemesplit}
\usepackage{hyperref}
\XeTeXlinebreaklocale "zh"
\XeTeXlinebreakskip = 0pt plus 1pt
\usepackage[cm-default]{fontspec}
\usepackage{xunicode,xltxtra}
\usepackage{indentfirst}
\usepackage{xeCJK}
\usepackage{graphicx}
\usepackage{algorithm}
\usepackage{algpseudocode}
\usepackage{amsmath}
\usepackage{xcolor}
\usepackage{listings}
\usepackage{fancyvrb}
\usepackage{tabularx}

\setsansfont{Kai} 

\XeTeXlinebreaklocale "zh"
\XeTeXlinebreakskip=0pt plus 1pt minus 0.1pt


\title
{基于GIT的分布式实时编译及BUG分析系统}
\subtitle{开题报告}
  \author{计92 陈宇恒}
\institute{清华大学计算机系}

\begin{document}
%-----------------------------------------------------------------------------80
  \frame
  {
    \titlepage
  }


  \frame{
	  \frametitle{系统功能}
	  本系统针对Linux内核基于GIT的分布式开发模式,实现以下功能:
	  \begin{itemize}
		  \item 实时编译测试,在一天之内对出现问题的
代码提交做出响应并通知对应开发者;
		\item 
对成功编译的版本,进行运行测试。对于Linux内核,此测试通过KVM运行编译出来的内核镜像,
捕捉启动过程和若干测试程序导致的Kernel Panic,进行分析并通知开发者;
		\item 在以上基础上,捕捉代码修改对软件性能的影响。
	  \end{itemize}

  }

\section*{大纲}
\frame{\tableofcontents }

  \section{系统架构}
  \frame{
	  \frametitle{单集群系统内部组成}
	本0Day测试系统可以部署在广域网的多个集群上,各个集群之间只共享少量信息。
	单集群系统主要由一下部件组成:
	\begin{itemize}
		\item GIT镜像服务器
		\item NFS服务器
		\item KVM测试调度器
	\end{itemize}

	另外,集群中的每个节点又可以担任(或同时担任)一下角色:
	\begin{itemize}
	\item 编译服务器
	\item KVM启动测试服务器
	\item LKP性能测试服务器
	\end{itemize}
	}

   \frame{
	   \frametitle{多集群系统组成}
	本系统具有自然的可扩展性,只需添加一个同步数据库即可推广到广域网上:
	\begin{itemize}
		\item 如上一小节所示的$n$个独立测试集群,之间通过广域网连接
		\item 一个中心的Key-Value数据库,我们使用GIT设计了一个分布式的Key-Value
			数据库以实现集群间数据同步的功能。
	\end{itemize}
	}

\section{毕设工作重点}
\begin{frame}[fragile]
	  \frametitle{构建基于GIT的Key-Value同步数据库}
	  不使用传统数据库的原因:避免数据库服务器的维护成本,
	  使用Github等免费服务即可。
	  基本算法:

\scriptsize{
\begin{Verbatim}[frame=single]
  while
    git pull --ff-only || {
      #should not error, try reset
      git reset --hard HEAD^
      continue
    }
    test_set keys,values

    git commit -a

    #if push success, finish
    git push && break

    #push failed, remote centeral repo updated
    git reset --hard HEAD^
  done
\end{Verbatim}
}

\end{frame}

  \frame{
	  \frametitle{Linux编译和运行时Bug分析}
	  \begin{itemize}
		  \item 静态分析部分 \\
	  现在使用的静态检查工具有:Cppcheck/Sparse/Coccinelle/Smatch。这部分工作
	  主要在分析系统已发现的10000+个错误/警告上。主要目标是减少这些分析工具产生
	  的False Positive。
	  	\item 运行是Bug分析 \\
		这部分主要集中在Kernel运行过程中产生的Panic信息分析上。在0Day系统
		运行过程中,已收集到3600多个Panic信息,我们将对这些信息进行分析
		整理,归纳其背后的规律。
	  \end{itemize}
  }

  \frame{
	  \frametitle{系统应用范围扩展}
	  一些推广思路:
  	\begin{itemize}
		\item 现在的系统重点针对内核,
将来应该可以比较容易地测试其他使用GIT开发模式的开源目标软件,比较容易加入其他测试工具。
		\item 实现一个通用模块,用来分析bug的一些共性信息,比如bug的life time,bug的特征(内存溢出,空指针,语义,并行...)


	\end{itemize}
  }

\section{相关工作}
\frame{
\frametitle{Fast 2013 Best Paper}
\emph{A Study of Linux File System Evolution}\\
\vspace{1em}
本文主要通过Linux开发过程中Git Commit中开发者给出的patch信息,分析
Linux Kernel文件子系统的代码演化。通过分析8年来Linux文件系统相关的
5079个patch,提出了一些新颖且令人惊讶的现象。这些对Patch分析的结果
非常有利于以后的文件系统开发和相关Bug自动分析工具的开发。

}

\frame{
	\frametitle{OSDI 2012}
\emph{Be Conservative: Enhancing Failure Diagnosis with Proactive Logging}\\
\vspace{1em}
这一论文更偏重与运行时错误的分析。
软件系统在实际运行中出错时,软件日志中的错误和警告信息经常是唯一可以用于分析定位
底层错误的手段。因此,这些Log的有效性就具有很强的重要性。但是,很少有人研究现有
日志信息的搜集方法和如何提高这些日志的有效性。本文针对五个广泛使用的大型开源软件做了
研究。从250个随机采样的系统错误中,发现有大半错误无法通过现有的日志信息很好地诊断
出来。但是另一方面,进一步分析显示,大多数这些错误可以分为几个常见的错误模式(例如
系统调用返回错误)。
}

\end{document}

